\documentclass[a4paper, 12pt]{article} % Fuente 12pt
\usepackage[utf8]{inputenc}
\usepackage[T1]{fontenc}
\usepackage{hyperref}
\usepackage[left=3cm, right=3cm, top=3.5cm, bottom=3.5cm]{geometry} % Márgenes recomendados
\usepackage{times} % Fuente Times New Romans
\usepackage[spanish, english]{babel} 
\usepackage[style=ieee, backend=bibtex,citestyle=numeric-comp]{biblatex} % Bibliografía en formato IEEE
\usepackage{sectsty}
\usepackage{cover/cover}
\usepackage{graphicx}
\graphicspath{ {images/} } % Directorio imágenes
\usepackage{listings} % Formateo código
\usepackage{scrextend}
\usepackage{longtable}
\usepackage{pgfplots} % Graficas de barras
\usepackage{pgf-pie} % Graficas de tartas
\usepackage{multirow} % multiples filas en las tablas
% Lista de acronimos
\usepackage[acronym]{glossaries}
\makeglossaries
% List of acronyms
\newacronym{CTF}{CTF}{Capture The Flag}
\newacronym{htb}{HTB}{Hack The Box}
\newacronym{ftp}{FTP}{File Transfer Protocol}
\newacronym{ssh}{SSH}{Secure Shell}
\newacronym{tcp}{TCP}{Transmission Control Protocol}
\newacronym{http}{HTTP}{Hypertext Transfer Protocol}
\newacronym{url}{URL}{Uniform Resource Locator}
\newacronym{ip}{IP}{Internet Protocol}
\newacronym{ids}{IDS}{Intrusion Detection System}
\newacronym{dns}{DNS}{Domain Name System}
\newacronym{uri}{URI}{Uniform Resource Identifier}
\sectionfont{\MakeUppercase} % Secciones en mayúsculas
\bibliography{sections/Bibliography.bib}

\Director{Luis Alfonso Rodríguez De Trio y Perez}
\Lugar{Madrid} 
\Master{Máster en Seguridad Informática y Hacking Ético} 
\Trabajo{TRABAJO FIN DE MÁSTER} 

\author{Víctor Nieves Sánchez}
\date{Octubre de 2021}
\title{Resolución y explicación de CTF\_X y CTF\_Y}

\begin{document}
\maketitle
\null
\newpage
\begin{otherlanguage}{spanish}
    
\pagenumbering{roman} % Numeración romana hasta la primera sección

%\begin{otherlanguage}{spanish}
%    \renewcommand{\spanishabstractname}{Agradecimientos}
%    \begin{abstract}
%        % Si lees esto es que eres o muy listo, o muy vago como para intentarlo.
% La solución es: 
%   1) Metadata
%   2) Comentario
%   3) Base64
%   4) JWT Decoder
%   5) Binary Decoder
%   6) Agradecimientos
%
%   No todo va a ser CTFs y hacer TFGs/TFMs ...

\begin{lstlisting}[language=html]
    <img src="./images/?.jpeg" alt="Let's play a game!" onerror=alert('WTF')>
\end{lstlisting}
%    \end{abstract}
%\end{otherlanguage}

\newpage
\tableofcontents

\newpage
\listoffigures

\newpage
\renewcommand{\listtablename}{Lista de Tablas}
\listoftables

\newpage
\lstlistoflistings

\newpage
\printglossary[type=\acronymtype]

\newpage
    \begin{abstract}
        \normalsize
        \# TODO

\textbf{Keywords:} Hacking, Ciberseguridad, \acrlong{HTB}, \acrshort{CTF}, \acrlong{ceh}\ldots

    \end{abstract}

\newpage
\begin{otherlanguage}{english}
    \renewcommand{\spanishabstractname}{Abstract}
    \begin{abstract}
        \normalsize
        In the technology industry, cybersecurity is very important.  Companies are increasingly focused on ensuring that their various applications, projects and services are secure.  They are even starting to adapt their solutions to technologies that guarantee greater security and robustness, such as Blockchian technology.\\

On the other hand, companies must protect the information security, guaranteeing the confidentiality, availability and integrity of the information. This problem is often linked to data protection regulations, such as the \acrshort{gdpr}, which is why the idea of a "\acrlong{ssi}" is becoming increasingly popular.  \acrfull{ssi} is the concept that individuals or organizations have the sole and exclusive ownership of their digital identities, and are the ones who control how their personal data is shared and used.\\

Alastria, a consortium promoting the digital economy through the development of decentralized registration technologies such as Blockchain, has defined a digital identity model called \textit{Alastria ID}. The \textit{Alastria ID} project is deployed as one of the basic applications of the blockchain infrastructure promoted by the consortium within its platform.  This technological proposal of digital identity in blockchain aims to provide an infrastructure and development framework to carry out \acrfull{ssi} projects, with full legal validity in the euro area.\\

The fact that a company uses Blockchain technology does not mean that it is free of security problems and vulnerabilities.  As these are new technologies, there are still not many standards, there are no robust implementations, and a small failure can cause a security gap, and the impact this can have on a \acrlong{ssi} model can be critical to the ecosystem.\\

The objective of this work is multiple. The concept of \acrfull{ssi} has been studied, as well as the implementation of Alastria called \textit{Alastria ID} and other implementations and solutions that exist today. On the other hand, a study of the \textit{Alastria ID} implementation has been carried out from the point of view of security, auditing the Smart Contracts designed in the \acrshort{mvp}1 and the library that facilitates its use. Finally, a \acrlong{poc} has been created for one of the vulnerabilities found during the analysis phase and its criticality and impact has been evaluated.\\

\textbf{Keywords:} Blockchain, Alastria, \acrlong{ssi}, Ethereum, Quorum, Solidity, Smart Contracts, Hacking, Cybersecurity, Information Security, Audit\ldots


    \end{abstract}
\end{otherlanguage}
\newpage
\pagenumbering{arabic} % Numeración árabe en la primera sección
\UseRawInputEncoding % UTF-8 Listings

\newpage
% \nocite{*} % Cita todas las ref (incluidas las no citadas)
\printbibliography[heading=bibnumbered] % Última sección, numerada, para la bibliografía

\end{otherlanguage}
\end{document}


