Constantemente son más comunes los ataques a las empresas y las organizaciones; cada vez son más comunes los ataques de \textit{ransomware}, los robos de información sensible o el \textit{phising}. En una sociedad cada vez más tecnológica como la actual, cada vez es más común que la tecnología esté a nuestro alrededor: móviles, ordenadores, elementos \acrshort{iot}, incluso hay tostadores inteligentes conectados en nuestras casas. Cada vez más y más dispositivos que trabajan con nuestros datos, conectados en nuestras redes, y esto, hace que el mundo sea un gran ``campo de juego`` para los \textit{hackers} malvados.\\

Es el objetivo de los \textit{hackers} buenos (\textit{white hats}) y de los expertos y profesionales de la ciberseguridad el proteger este mundo tecnológico. Desde el aplicativo privado del banco, hasta la aplicación más popular del móvil. Todo elemento que vive en la red, puede ser atacado de mil maneras: \textit{phising}, \textit{man-in-the-middle}, \textit{\acrshort{ddos}}, \ldots por eso es fundamental que los defensores tengan la capacidad de pensar como atacantes, para poder crear las defensas pertinentes.\\

Es aquí donde el \textit{hacking ético} entra en acción. Un \textit{hacker ético} es un profesional con las habilidades de penetrar en redes, explotar vulnerabilidades y acceder a las máquinas a placer. Los \textit{hackers} usan diversas herramientas y conocimientos, además de estar siempre al día de las últimas vulnerabilidades y \acrshort{cve}s.\\

Personalmente, llevo un tiempo indagando en el mundo de la ciberseguridad y el hacking, por eso he elegido este tema para mi Trabajo de Fin De Máster. El objetivo de este documento es mostrar mi manera de pensar y mi toma de decisiones a la hora de resolver dos máquinas en \acrlong{HTB}, explicando las herramientas usadas y las conclusiones obtenidas. Este documento marcará mis inicios en el mundo de la ciberseguridad. Mundo en el que espero trabajar profesionalmente dentro de poco.