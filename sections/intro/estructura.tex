Este documento se ha dividido en los siguientes capítulos para poder realizar los objetivos anteriormente citados.\\

El segundo capitulo \textit{Estado del Arte}, contiene una introducción al término de \textit{hacking ético}, enumerando los distintos tipos de \textit{hacker} que hay y las habilidades y cualidades que presenta un \textit{hacker}. Finalmente se hablará de la plataforma \acrlong{HTB}, la cual ha sido la elegida para realizar este trabajo. Se hablará de sus características y de las distintas modalidades que contiene.\\

El siguiente capitulo, \textit{Herramientas} enumera todas las herramientas utilizadas en los capítulos 4 y 5. En cada apartado habrá una pequeña descripción de la herramienta, así como los comandos para su ejecución y figuras que faciliten su comprensión.\\

El capítulo 4, \textit{Máquina 1: Cap}, contiene la primera máquina realizada, llamada \textit{Cap}. En este capitulo se podrán encontrar las fases principales de la metodología hacking que son aplicables a un programa de ``capturar la bandera`` (\acrlong{CTF}). Se podrá encontrar los pasos y las decisiones tomadas en las fases de reconocimiento, enumeración, ganar acceso y escalada de privilegios. Por último se encontrará un resumen de las vulnerabilidades encontradas en la máquina así como algunos comentarios y pensamientos obtenidos tras la realización de la máquina.\\

Tras la primera máquina, se realizará el estudio de la segunda. El capitulo \textit{Máquina 2: Seal} contiene, al igual que el capitulo previo, los pasos realizados para lograr completar la máquina, al igual que un apartado de resumen con el objetivo de resumir la máquina y comentar algunos aspectos relevantes.\\

El penúltimo capítulo, \textit{Conclusiones}, recoge los resultados de todo el trabajo y justifica la satisfacción de los objetivos mencionados en el capitulo anterior.\\

Por último, en el capítulo \textit{Trabajo Futuro} se comentarán las líneas futuras de trabajo, y las intenciones personales tras la realización de este documento.
