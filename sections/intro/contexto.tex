\subsubsection{Ciberseguridad}
La seguridad de la información o \textit{ciberseguridad} se refiere a la protección o salvaguarda de la información y de los sistemas de información que utilizan, almacenan y transmiten información. La información es un activo crítico que las organizaciones deben asegurar. Si la información sensible cae en manos equivocadas, la organización respectiva puede sufrir enormes pérdidas en términos económicos, de reputación, de clientes, o de otras maneras.\\

El elementos principales de la seguridad de la información son:
\begin{itemize}
    \item \textbf{Confidencialidad}. La confidencialidad es la garantía de que la información es accesible sólo para los autorizados. Las violaciones de la confidencialidad pueden producirse debido a una manipulación inadecuada de los datos o a un intento de piratería informática. Los controles de confidencialidad incluyen la clasificación de los datos, el cifrado de los mismos y la eliminación adecuada de los equipos.
    \item \textbf{Integridad}. La integridad es la fiabilidad de los datos o recursos en la prevención de cambios indebidos y no autorizados; la garantía de que la información es suficientemente precisa para su propósito. Las medidas para mantener la integridad de los datos pueden incluir una suma de comprobación o \textit{checksum} (un número producido por una función matemática para verificar que un determinado bloque de datos no ha sido modificado) y el control de acceso (que garantiza que sólo las personas autorizadas pueden actualizar, añadir o eliminar datos).
    \item \textbf{Disponibilidad}. La disponibilidad es la garantía de que los sistemas responsables de entregar, almacenar y procesar la información son accesibles cuando los usuarios autorizados lo requieren. Las medidas para mantener la disponibilidad de los datos pueden incluir matrices de discos para sistemas redundantes y máquinas agrupadas, software antivirus para combatir el malware y sistemas de prevención de la denegación de servicio distribuida (\acrshort{ddos}).
    \item \textbf{Autenticidad}. La autenticidad se refiere a la característica de la comunicación, los documentos o cualquier dato que garantice la calidad o que no se ha corrompido. La función principal de la autenticación es confirmar que un usuario es auténtico. Controles como la biometría, las tarjetas inteligentes y los certificados digitales garantizan la autenticidad de los datos, las transacciones, las comunicaciones y los documentos.
    \item \textbf{No repudio}. El no repudio es una forma de garantizar que el remitente de un mensaje no pueda negar posteriormente haber enviado el mensaje y que el destinatario no pueda negar haber recibido el mensaje. Los particulares y las organizaciones utilizan las firmas digitales para garantizar el no repudio
\end{itemize}

La ciberseguridad no solo se centra en la seguridad de la información. Algunas categorías comunes agrupadas en la ciberseguridad son:
\begin{itemize}
    \item La seguridad de red.
    \item La seguridad de las aplicaciones.
    \item La seguridad operativa.
    \item La recuperación ante desastres y la continuidad del negocio.
\end{itemize}

\subsubsection{Hacking}

El \textit{hacking} es el conjunto de técnicas a través de las cuales se accede a un sistema informático vulnerando las medidas de seguridad establecidas originariamente. Por lo general, cuando se habla de \textit{hacking} se suele hacer alusión a un acceso ilícito; pero el hacking como tal, no es más que un conjunto de técnicas utilizadas para introducirse en un sistema informático vulnerando las medidas de seguridad, con independencia de la finalidad con la cual se realice, puede ser lícito y solicitado.\\

En el capitulo \textit{Hacking Ético} se hablará más en detalle sobre lo que es el \textit{hacking ético}, los tipos de \textit{hacker}, así como algunos de los términos más ligados a la palabra ``\textit{hacking}`` y las cualidades que tiene un \textit{hacker}.