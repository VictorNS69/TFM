El programa \textit{\acrfull{ceh}} define el hacking ético como:

\begin{quote}
    \textit{la práctica de emplear conocimientos informáticos y de redes para ayudar a las organizaciones a probar la seguridad de su red en busca de posibles lagunas y vulnerabilidades}
\end{quote}

Los hackers éticos informan de todas las vulnerabilidades al propietario del sistema y de la red para que se corrijan, aumentando así la seguridad del sistema de información de una organización. El hacking ético implica el uso de herramientas, trucos y técnicas de hacking típicamente utilizadas por un atacante para verificar la existencia de vulnerabilidades explotables en la seguridad del sistema.\\

Hoy en día, el término \textit{hacking} está estrechamente asociado a actividades ilegales y poco éticas. Se sigue debatiendo si el hacking puede ser ético o no, dado que el acceso no autorizado a cualquier sistema es un delito. Por eso es importante destacar las siguientes definiciones:

\begin{itemize}
    \item El término ``\textbf{hacker}`` se refiere a una persona que disfruta aprendiendo los detalles de los sistemas informáticos y ampliando sus capacidades.
    \item El verbo ``\textbf{hackear}`` describe el rápido desarrollo de nuevos programas o la ingeniería inversa del software existente para hacerlo mejor o más eficiente de formas nuevas e innovadoras.
    \item Los términos ``\textbf{cracker}`` y ``\textbf{atacante}`` se refieren a las personas que emplean sus habilidades de hacking con fines ofensivos.
    \item El término ``\textbf{hacker ético}`` se refiere a los profesionales de la seguridad que emplean sus habilidades de hacking con fines defensivos.
\end{itemize}

En la actualidad, el hacking ético es necesario porque permite a las organizaciones contrarrestar los ataques de los hackers malintencionados, anticipando los métodos que utilizan para entrar en los sistemas. El hacking ético ayuda a predecir con antelación las distintas vulnerabilidades posibles y a rectificarlas antes de sufrir ningún tipo de ataque externo.


\subsubsection{Tipos de Hacker}

Aunque solo existe una tipología de hacking que sea ético, es interesante explicar los tipos de hacker que hay.

\begin{itemize}
    \item \textbf{Sombrero negro} o \textbf{Black hat}: son aquellos que tiene los conocimientos necesarios para romper la seguridad de un sistema o programa, o para crear virus y los utiliza con malas intenciones.
    \item \textbf{Sombrero blanco} o \textbf{White hat}: cuentan con los conocimientos necesarios para saltarse la seguridad de un sistema o crear virus, pero los usan para hacer el bien ayudando a mejorar la seguridad de organizaciones y empresas.
    \item \textbf{Sombrero gris} o \textbf{Grey hat}: son aquellos que tienen una ética ambigua. Es decir, tienen los conocimientos de los \textit{blacks hats} y los utilizan para entrar en sistemas pero no explotan las vulnerabilidades con fines maliciosos.
\end{itemize}

También existen otros términos para definir a los hackers, como los \textbf{\textit{script kiddies}}, los \textbf{\textit{hacktivistas}} o los equipos/grupos de hackers (\textbf{\textit{hacker teams}}).

\subsubsection{Habilidades de un Hacker Ético}

Es esencial que un hacker ético adquiera los conocimientos y habilidades necesarios para convertirse en un hacker experto y utilizar estos conocimientos de forma lícita. Algunas de esas habilidades son:

\begin{itemize}
    \item \textbf{Habilidades técnicas}:
          \begin{itemize}
              \item Conocimiento profundo de los principales entornos operativos, como \textit{Windows}, \textit{Linux/Unix} y \textit{Mac OS}.
              \item Conocimiento profundo de los conceptos y tecnologías de redes, y del hardware y software relacionados.
              \item Conocimiento de las áreas de seguridad y temas relacionados.
              \item Alto conocimiento técnico de cómo lanzar ataques sofisticados.
          \end{itemize}
    \item \textbf{Habilidades no técnicas}:
          \begin{itemize}
              \item La capacidad de aprender y adaptar rápidamente las nuevas tecnologías.
              \item Una fuerte ética de trabajo y buenas habilidades de comunicación y resolución de problemas.
              \item Compromiso con las políticas de seguridad de la organización.
              \item Conocimiento de las normas y leyes locales.
          \end{itemize}
\end{itemize}