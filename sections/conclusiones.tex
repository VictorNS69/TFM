En una sociedad tan dependiente de la tecnología como la actual, son cada vez más comunes los ataques: las estafas electrónicas, los robos de información o los ataques a las empresas con \textit{ransomware}. Estos ataques los hacen los conocidos \textit{hackers}, pero como ya hemos visto, un ``\textit{hacker}`` no es siempre alguien con el objetivo de hacer el mal. Los \textit{hackers}, en concreto los \textit{white hats} ayudan y mejoran la seguridad en la red; ayudan a las empresas frente a los ataques, y a fortalecer sus aplicaciones y servicios en la red.\\

Uno de los principales elementos que hace a un \textit{hacker} ser bueno es la experiencia obtenida con los años. Por eso, programas como \acrfull{HTB} permiten a los profesionales mejorar sus habilidades prácticas. Además de \acrshort{HTB}, existen otras herramientas en la red, como los desafíos de \acrlong{CTF} o los programas de \textit{Bug Bounty}. Con los dos primeros mencionados, es posible elegir temática y dificultad, teniendo la posibilidad de utilizar estas herramientas siendo alguien nuevo en la materia, o un \textit{hacker} experto. Respecto a los \textit{bug bounties}, requiere algo más de experiencia, ya que los programas no son otra cosa que empresas que publican sus servicios (como páginas web) a disposición de los \textit{hackers}, con la intención de que estos encuentren vulnerabilidades a cambio de recompensas económicas, así la empresa puede corregir las fallas antes de que un atacante con malas intenciones las explote.\\

En el estudio realizado, se han mostrado la resolución de dos máquinas, \textit{Cap} y \textit{Seal}, de dificultades \textit{easy} y \textit{medium} respectivamente. Se han utilizado diversas herramientas y técnicas, consiguiendo en ambas máquinas, el objetivo final: obtener el flag de \textit{root}. Al elegir dos máquinas, se ha podido estudiar dos sistemas completamente distintos. Ambas máquinas tenían un servicio web con el cual se ha logrado explotar una vulnerabilidad y entrar al sistema. Una vez en el sistema, se ha realizado escalada de privilegios con distintas técnicas.\\

Este trabajo ha sido muy instructivo y divertido, ya que la mayor parte del trabajo ha sido práctico y de investigación. Se ha tenido que investigar mucho, tanto las tecnologías usadas, como las versiones de los servicios desplegados (con la intención de encontrar \acrshort{cve}s). En conclusión, recomiendo el uso de este tipo de herramientas, como \acrshort{HTB}, ya que permiten practicar los conocimientos y habilidades de \textit{hacking}, además de realizarse a modo de ``\textit{juego}``.\\

Por último, este trabajo también ha servido para estudiar lo que supone cualquier tipo de servicio para una entidad. Un simple fallo de configuración, un permiso mal puesto en un archivo o una tecnología no actualizada puede derivar en una gran brecha de seguridad para la empresa, permitiendo la obtención de datos a usuarios no autorizados, la ejecución de código en un servidor, o permitir una conexión con los elementos internos y privados de la empresa.