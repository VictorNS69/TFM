En la industria tecnológica,  es muy importante la ciberseguridad. Las empresas cada vez más se centran en garantizar que sus distintas aplicaciones, proyectos y servicios son seguros; que estén protegidos de amenazas internas, externas y vulnerabilidades.\\

Por otro lado, al tener cada vez más y más dispositivos inteligentes conectados a la red, aplicaciones móviles, portales webs y servicios en la red entre otros, proporciona a los hackers un campo de juego enorme, y un escenario muy llamativo para hacer maldades. Por suerte, no todos los hackers son malos. Los hackers éticos, aquellos que usan sus habilidades para mejorar la seguridad en la red son cada vez más aclamados por las empresas.\\

Con el objetivo de introducirme en el mundo de la ciberseguridad y el hacking, nace este trabajo. Como ``novato`` en la materia, me he encontrado muchas herramientas y artículos que leer, que me permitan aprender y mejorar mis conocimientos y habilidades. Una de las herramientas más divertidas, cómodas y sencillas (de utilizar) que he encontrado es \acrlong{HTB} y los eventos y desafíos estilo \acrlong{CTF}. Estos desafíos permiten aumentar los conocimientos en distintas tecnologías, como web, móvil o en ciertos campos como la criptografía o el reversing, a base ejemplos de vulnerabilidades que se pueden encontrar en el mundo real. \\

En este documento se explicará la plataforma de \acrlong{HTB} y se resolverán dos máquinas de la plataforma; se compartirán los descubrimientos y resultados obtenidos, así de los motivos tras las decisiones tomadas.\\

\textbf{Keywords:} Hacking, Hacking ético, Ciberseguridad, \acrlong{HTB}, \acrlong{CTF}, \acrlong{ceh}\ldots
